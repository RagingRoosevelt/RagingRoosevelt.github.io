\documentclass{report}

\usepackage[fleqn]{amsmath}
\usepackage{multicol}
\usepackage{enumitem}
\usepackage{amsfonts}

\usepackage[top=0.3in, bottom=0.3in, left=0.3in, right=0.3in]{geometry}

\begin{document}
\setlength{\columnsep}{3cm}
\begin{multicols}{2}
\setlength{\columnseprule}{.4pt}

\begin{itemize}[leftmargin=.25cm]
\item Area between curves (ch 6.1)

(provided that on the interval $[a,b]$, $ f(u) \geq g(u) $
\begin{align*}
A = \int_a^b [ f(u) - g(u) ] \ du
\end{align*}
Essentially, $ f(u) - g(u) $ needs to be $ \left\{ \textrm{right function} \right\} - \left\{ \textrm{left function} \right\} $ if the functions are in terms of $ y $ and you have $ dy $.

Otherwise, $ f(u) - g(u) $ needs to be $ \left\{ \textrm{upper function} \right\} - \left\{ \textrm{lower function} \right\} $ if the functions are in terms of $ x $ and you have $ dx $.




\item Volume (ch 6.2, 6.3)
\begin{itemize}[leftmargin=.25cm]
\item General:
\begin{align*} 
V(u) = \int_{u=a}^{u=b} A(u) \ du \hfill \\
\end{align*}

\item Disk Method
\begin{align*}
&\textrm{rotation about x-axis:} \hspace{0.25cm} V(x) = \int_{x=a}^{x=b} \pi \left( f(x) \right)^2 \ dx \hfill \\
&\textrm{rotation about y-axis:} \hspace{0.25cm} V(y) = \int_{y=a}^{y=b} \pi \left( g(y) \right)^2 \ dy \hfill \\
\end{align*}

\item Cylindrical shells
\begin{align*}
&\textrm{rotation about y-axis:} \hspace{0.25cm} V(x) = \int_{x=a}^{x=b} \left( 2 \pi x \right) \left( f(x) \right) \ dx \\
&\textrm{rotation about x-axis:} \hspace{0.25cm} V(y) = \int_{y=a}^{y=b} \left( 2 \pi y \right) \left( g(y) \right) \ dy
\end{align*}
\end{itemize}




\item Arc Length (ch 6.4)
\begin{itemize}[leftmargin=.25cm]
\item If $ x = f(t) $ and $ y = g(t) $
\begin{align*}
L = \int_{t=a}^{t=b} \sqrt{ \left( \frac{dx}{dt} \right)^2 + \left( \frac{dy}{dt} \right)^2 } \ dt
\end{align*}

\item If $ x = x $ and $ y = g(x) $
\begin{align*}
L = \int_{x=a}^{x=b} \sqrt{ 1 + \left( \frac{dy}{dx} \right)^2 } \ dx
\end{align*}

\item If $ x = f(y) $ and $ y = y $
\begin{align*}
L = \int_{y=a}^{y=b} \sqrt{ \left( \frac{dx}{dy} \right)^2 + 1 } \ dy
\end{align*}
\end{itemize}




\item Average Value (ch 6.5)
\begin{align*}
f = \frac{1}{b-a} \int_a^b f(x) \ dx
\end{align*}




\item Work (ch 6.6)
\begin{align*}
W = \int_a^b f_{orce} (x) \ dx
\end{align*}




\item Center of Mass (ch 6.6)
\begin{itemize}[leftmargin=.25cm]
\item Moments
\begin{align*}
M_y &= \rho \int_a^b x f(x) \ dx \\
M_x &= \rho \int_a^b \frac{1}{2} \left[ f(x) \right]^2 \ dx
\end{align*}
\item Centroid
\begin{align*}
\bar{x} &= M_y / \left( \int_a^b f(x) \ dx \right) \\
&= \frac{1}{A} \int_a^b x f(x) \ dx \\
\bar{y} &= M_x / \left( \int_a^b f(x) \ dx \right) \\
&= \frac{1}{A} \int_a^b \frac{1}{2} \left[ f(x) \right]^2 \ dx
\end{align*}
\end{itemize}




\item Economic Surplus (ch 6.7)
\begin{itemize}[leftmargin=.25cm]
\item Consumer Surplus

Given a ``Production Level'' of $ C $, then $ P = p(C) $ and Consumer Surplus is
\begin{align*}
\int_0^C \left[ p(x) - P \right] \ dx
\end{align*}


\item Producer Surplus

Given a ``Production Level'' of $ C $, then $ P = p(C) $ and Producer Surplus is
\begin{align*}
\int_0^C \left[ P - p(x) \right] \ dx
\end{align*}
\end{itemize}




\item Probability (ch 6.8)
\begin{itemize}[leftmargin=.25cm]
\item Probability Density Function

$ f(x) $ is a probability density function if both of the following are true:
\begin{itemize}[leftmargin=.25cm]
\item $ f(x) \geq 0 $ for all $ x $
\item $ 1 = \int_{-\infty}^{\infty} f(x) \ dx $


\end{itemize}
\item Probability of an event

If $ f(x) $ is a probability density function, then
\begin{align*}
\mathbb{P} \left( a \leq X \leq b \right) = \int_a^b f(x) \ dx
\end{align*}


\item Average Value (Mean)

if $ f(x) $ is a probability density function, then the mean or average value is given by
\begin{align*}
\mu = \int_{-\infty}^{\infty} x f(x) \ dx
\end{align*}
\end{itemize}





\end{itemize}
\vfill
\end{multicols}{2}
\end{document}