\documentclass[10pt]{amsart}


%------------------------------------%
% packages and formatting  %
%------------------------------------%

\usepackage{amsmath}
\usepackage{amssymb}
\usepackage{anysize}
\usepackage{float}
\usepackage{graphicx}
\usepackage{mathrsfs}
\usepackage{wasysym}
\usepackage{hyperref}

% margin size {left}{right}{top}{bottom}
\marginsize{.9in}{.9in}{.6in}{.6in}

% paragraph indentation
\setlength{\parindent}{0in}


%-----------------------------%
%  custom commands  %
%-----------------------------%

% derivatives, integrals, and sums
\newcommand{\pd}[3] {\ensuremath{ \frac{\partial^{#3} #1}{\partial #2^{#3}} }}
	% \pd{function}{variable}{partial #}   ---  partial derivative
\newcommand{\Int}[4] {\ensuremath{ \int_{#3}^{#4} #1 \ d #2 }}
	% \Int{integrand}{differential}{lower bound}{upper bound}
\newcommand{\Sum}[3] {\ensuremath{ \sum_{#2}^{#3} #1 }}
	% \Sum{summand}{lower bound}{upper bound}

% trigonometric and logarithmic functions
\newcommand{\Cos}[2] {\ensuremath{ \cos^{#2} \left( #1 \right) }}
	% \Cos{parameter}  ---  cos with parentheses
\newcommand{\Sin}[2] {\ensuremath{ \sin^{#2} \left( #1 \right) }}
	% \Sin{parameter}  ---  sin with parentheses
\newcommand{\Tan}[2] {\ensuremath{ \tan^{#2} \left( #1 \right) }}
	% \Tan{parameter}  ---  tan with parentheses
\newcommand{\Log}[1] {\ensuremath{ \log \left( #1 \right) }}
	% \Log{parameter}  ---  log with parentheses
\newcommand{\Ln}[1] {\ensuremath{ \ln \left( #1 \right) }}
	% \Ln{parameter}  ---  ln with parentheses
	
% piecewise functions
 \newcommand{\pwtwo}[5] {\ensuremath{ #1 = \left\{ \begin{array}{ll} #2, & \ \textrm{if } \ #3 \\ #4, & \ \textrm{if } \ #5 \end{array} \right. }}
 	% \pwtwo{function}{value #1}{range #1}{value #2}{range #2}  ---  piecewise function with two parts
\newcommand{\pwthree}[7] {\ensuremath{ #1 = \left\{ \begin{array}{ll} #2, & \ \textrm{if } \ #3 \\ #4, & \ \textrm{if } \ #5  \\ #6, & \ \textrm{if } \ #7 \end{array} \right. }}
 	% \pwthree{function}{value #1}{range #1}{value #2}{range #2}{value #3}{range #3}  ---  piecewise function with two parts

% graphics
\newcommand{\image}[4] {\begin{figure}[H] \centering \includegraphics[width=#2]{#1} \caption{\tiny{#3}} \label{#4} \end{figure}}
	% \image{filename}{size cm}{caption}{label}  ---  insert centered image with caption and label
	
% limits
\newcommand{\Lim}[2] {\ensuremath{ \lim_{#1 \rightarrow #2} }}
	% \limp{variable}  ---  limit as [variable] goes to [number]
	
% matrices and determinants
\newcommand{\matrixtwo}[4] {\ensuremath{ \begin{bmatrix} #1 & #2 \\ #3 & #4 \end{bmatrix} }}
	% \matrixtwo{11}{12}{21}{22}  ---  two-by-two matrix
\newcommand{\matrixthree}[9] {\ensuremath{ \begin{bmatrix} #1 & #2 & #3 \\ #4 & #5 & #6 \\ #7 & #8 & #9 \end{bmatrix} }}
	% \matrixthree{11}{12}{13}{21}{22}{23}{31}{32}{33}  ---  three-by-three matrix
\newcommand{\dettwo}[4] {\ensuremath{ \left| \begin{array}{cc} #1 & #2 \\ #3 & #4 \end{array} \right| }}
	% \dettwo{11}{12}{21}{22}  ---  two-by-two determinant
\newcommand{\detthree}[9] {\ensuremath{ \left| \begin{array}{ccc} #1 & #2 & #3 \\ #4 & #5 & #6 \\ #7 & #8 & #9 \end{array} \right| }}
	% \detthree{11}{12}{13}{21}{22}{23}{31}{32}{33}  ---  three-by-three determinant
	
% number systems
\newcommand{\reals} {\ensuremath{ \mathbb{R} }}
\newcommand{\complexes} {\ensuremath{ \mathbb{C} }}


%--------------------%
%  title / header  %
%--------------------%

\title{Title}
\author{Author}


%--------------------------%
%  document proper  %
%--------------------------%

\begin{document}

{\bf Math 121: Calculus} \hfill {\bf Syllabus} \hfill {\bf Spring 2014}

\hrulefill 

\vspace{.5cm}

{\bf Instructor:} Theodore Lindsey 	\hfill  {\bf Coordinator:} Milena Stanislavova \hspace{.15cm} \\
{\bf Office:} Snow 543	\hfill  {\bf Office:} 525 Snow Hall  \hspace{2.2cm} \\
{\bf Hours:} MWR 12-1 	\hfill   \\
{\bf Email:} \href{mailto:tlindsey@math.ku.edu}{tlindsey@math.ku.edu}	\hfill  {\bf Email:} \href{mailto:stanis@math.ku.edu}{stanis@math.ku.edu} \hspace{1.25cm} \\



\,\

{\bf Course Information:}
\begin{itemize}
	\item {\bf Time:} MTWRF 11:00 - 11:50 p.m. 
	\item {\bf Room:} Snow 306 (may change after a week or so, will be announced in class, via email, and on classroom blackboard)
	\item {\bf Line \#:} 59424 (lecture) and 59428 (lab)
	\item {\bf Description:} Differentiation and integration of algebraic and trigonometric functions. Applications to physical sciences and engineering.
\end{itemize}

\,\,

{\bf Required Materials:}
\begin{itemize}
	\item {\bf Text:} WebAssign + eBook access for Stewart's Calculus, ISBN: 9780538738071. 
	\item {\bf Tech:} graphing calculator or phone/tablet/laptop with access to math software or web-resources. 
\end{itemize}

\,\,

{\bf Exams:}
\begin{itemize}
	\item There will be four exams:
	\begin{itemize}
		\item {\bf Exam 1 \& 3}, worth {\bf 10\%} each of the final grade, will be in class (the week of February 17th and the week of April 17th).
		\item {\bf Midterm Exam}, worth {\bf 20\%} of the final grade, will be a common exam (March 12th, 5:45 to 7:45 pm).
		\item {\bf Final Exam}, worth {\bf 30\%} of the final grade, will be a common exam (May 12th, 4:30 to 7 pm).
	\end{itemize}
	\item {\bf If a student has a preexisting conflict with the original scheduled time for any exam, he/she must make arrangements with the course coordinator in advance.}  If a student misses  an exam due to an emergency, he/she must contact the coordinator immediately following the incident in order to schedule a make up exam.  In either case, justifying documentation will be required.
	\item A student must bring writing implements and photo identification to each exam.
\end{itemize}

\,\,

{\bf Evaluation \& Point Distribution:}
\begin{itemize}
	\item The course is divided into four units, with an exam marking the conclusion of each.
	\item A student's accumulation of points throughout all units will be the benchmark for the final grade he/she receives in the course. 
	\item Points may be earned from homework, quizzes, the Gateway Exam, attendance/participation, and exams according to the following distribution (letter grades will be assigned on 10\% increments):
\end{itemize}

\,\,

\begin{center}
\begin{tabular}{| l | r |}
\hline
Homework \hspace{1cm} & 100  \\
Quizzes & 50 \\
Gateway Exam & 100 \\
Attendance & 50 \\
Exam 1 & 100  \\
Midterm Exam & 200  \\
Exam 3 & 100  \\
Final Exam & 300  \\
\hline
Total  & 1000 \\
\hline
  \end{tabular}
\end{center}

\,\,

{\bf Homework \& Quizzes:}
\begin{itemize}
	\item Homework will be assigned and collected weekly, both online and in class.
	\item Quizzes may be given occasionally, both online and in class.
	\item See the course webpage for more details about homework \& quizzes.
	\item Online homework will always be due at midnight on Saturday of the week it is given.
	\item Traditional homework will be collected during the first meeting of the week
\end{itemize}

\,\,

{\bf Attendance \& Participation:}
\begin{itemize}
	\item Class attendance and participation are vital components of student success.
	\item Attendance points comprise a significant percentage of the final grade and should not be overlooked.
\end{itemize}

\,\,

{\bf Course Structure:}
\begin{itemize}
	\item The course will be comprised of a series of ``mini-units" focusing on fundamental Calculus topics.  
	\item Each mini-unit will involve a motivating exploration of one or more relevant, real-world problems, followed by a sequential development of the mathematical tools needed to solve them.  
	\item Once this computational and theoretical framework is in place, the original problems will be revisited and related problems will be considered.
\end{itemize}

\,\,

{\bf Academic Honesty:}
\begin{itemize}
	\item There is a zero-tolerance policy for cheating, plagiarism, of any other forms of academic misconduct.  
\end{itemize}

\,\,

{\bf Office Hours and Help Room:}
\begin{itemize}
	\item Students are encouraged to bring questions to the instructor's office hours.
	\item If a student has unavoidable conflicts with all posted hours, he/she may request an individual appointment.
	\item The Calculus Resource Center, located in 439 Anschutz Library, is a drop-in help room for Math 121/122 students (the operating hours are MTWRF 10am to 4pm and
MTWR 7pm to 9 pm. ).
\end{itemize}

\,\,

{\bf Email Communication:}
\begin{itemize}
	\item The instructor will periodically disperse course announcements via email (and, simultaneously, via Blackboard announcements).
	\item Students are responsible for checking their email daily in order to be aware of all such announcements.
	\item The email address a student has filed with the Registrar's Office will be used unless he/she makes other arrangements with the instructor.
\end{itemize}

\,\,

{\bf Prerequisites:}
\begin{itemize}
    \item Note: Math 121 is open for only two credits to students with credit in Math 115.
    \item To encourage success in Math 121, students are expected to have completed one of the following:
    \begin{itemize}
    	\item[$\circ$] Math 103 or Math 104,
    	\item[$\circ$] three years of college preparatory mathematics, including trigonometry, and a score of 28 or more on Enhanced ACT Mathematics,
    	\item[$\circ$] a qualifying score on the mathematics placement test.
    \end{itemize}
\end{itemize}

\,\,

{\bf Add/Change Information:}
\begin{itemize}
	\item Students should meet with Lindsey Deaver (\href{mailto:ldeaver@math.ku.edu}{ldeaver@math.ku.edu}), the KU mathematics enrollment advisor, to discuss any add or change of sections after the online period allowed by the university.
\end{itemize}

\,\,

{\bf Drop Information:}
\begin{itemize}
	\item The last day to drop a Spring 2014 full-semester KU course without penalty is Monday, February 10.
	\item The last day to withdraw from a Spring 2014 full-semester KU course is Monday, April 21.
\end{itemize}

\,\,


{\bf Special Needs:}
\begin{itemize}
	\item The staff of \href{http://disability.ku.edu/}{Accessibility Resources} in 22 Strong Hall (785-864-2620) coordinates accommodations for students in all KU courses. 
	\item Students should bring any special needs to the attention of the instructor within two weeks of the first day of classes.
\end{itemize}

\,\,




\end{document}




%--------------------------%
%  optional symbols  %
%--------------------------%

% cursive fonts
	% \mathcal{letter}
	% \mathscr{letter}

% blank space
	% \hspace{size cm}
	% \vspace{size cm}
	% \skip, \bigskip, etc.

% square root symbol with "tail"
	\comment{ 
	\let\oldsqrt\sqrt
		% rename sqrt as oldsqrt
 	\def\sqrt{\mathpalette\DHLhksqrt}
	\def\DHLhksqrt#1#2{
	\setbox0=\hbox{$#1\oldsqrt{#2\,}$}\dimen0=\ht0
	\advance\dimen0-0.2\ht0
	\setbox2=\hbox{\vrule height\ht0 depth -\dimen0}
	{\box0\lower0.4pt\box2}}
	}

